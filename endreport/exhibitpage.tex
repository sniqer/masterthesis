\documentclass [cropmarks, frame, english, master]{idaexhibitpage}
\usepackage [latin1]{inputenc}
\isbn {ISBN}
\author {Paul Nedstrand \& Razmus Lindgren}
\thesisnumber {THESISNUMBER}
\titleswedish {Svensken Titel}
\titleenglish {Test Data Post-Processing and Analysis of LA \& HARQ}
\degreetype {teknologie}
\degreesubject {Engineering}
\degreesubjectswedish {DEGREESUBJECTSWEDISH}
\presentationplace {PRESENTATIONPLACE}
\presentationhouse {PRESENTATIONHOUSE}
\presentationhus {PRESENTATIONHUS}
\presentationdate {PRESENTATIONDATE}
\presentationdatum {PRESENTATIONDATUM}
\presentationtime {PRESENTATIONTIME}
\keywords {KEYWORDS}
\nyckelord {NYCKELORD}
\facultyexaminername {OPPONENT NAME}
\facultyexaminertitle {OPPONENT TITEL}
\facultyexamineraddress {OPPONENT ADDRESS}
\thesisurl {http://XXX}
\supportedby {SUPPORTEDBY}
\newcommand {\issn }{-}
\begin {document}
\exhibitpagebeforeabstract 
\S  \ This Master thesis involves developing a lightweight analyser that produce statistics from the communication between a cell phone and a E-UTRAN base station. The analyser tool will produce graphs with information about the correlation between a signal throughput and the interference in the channel that the signal is sent through. From the statistics produced by the analyser tool, the testing personal at Ericsson can more easily deduce where the interference in signals arises from. 
\exhibitpageafterabstract 
\end {document}
