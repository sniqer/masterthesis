\documentclass [cropmarks, frame, english, master]{idaexhibitpage}
\usepackage [latin1]{inputenc}
\isbn {ISBN}
\author {Paul Nedstrand \& Razmus Lindgren}
\thesisnumber {THESISNUMBER}
\titleswedish {Svensken Titel}
\titleenglish {Test Data Post-Processing and Analysis of LA}
\degreetype {teknologie}
\degreesubject {Engineering}
\degreesubjectswedish {DEGREESUBJECTSWEDISH}
\presentationplace {PRESENTATIONPLACE}
\presentationhouse {PRESENTATIONHOUSE}
\presentationhus {PRESENTATIONHUS}
\presentationdate {PRESENTATIONDATE}
\presentationdatum {PRESENTATIONDATUM}
\presentationtime {PRESENTATIONTIME}
\keywords {KEYWORDS}
\nyckelord {NYCKELORD}
\facultyexaminername {OPPONENT NAME}
\facultyexaminertitle {OPPONENT TITEL}
\facultyexamineraddress {OPPONENT ADDRESS}
\thesisurl {http://XXX}
\supportedby {SUPPORTEDBY}
\newcommand {\issn }{-}
\begin {document}
\exhibitpagebeforeabstract 
 \S  \ This master thesis involves developing a lightweight analysis tool that produce statistics in form of graphs from the traffic data in the communication link between a UE, e.g. a cell phone, and a base station which the cell phone is connected to. The analysis tool will produce graphs with information about the correlation between two signal data in the channel (e.g throughput over interference). From the statistics produced by the analysis tool, the testing personnel at Ericsson can easily detect potential faulty behaviour from the UE or eNodeB in a more exact way than they were able to do before. This tool will also be able to help Ericsson rewrite test cases from being pretty basic to cover a larger area. The tool will mainly be oriented on analysing link adaptation and HARQ. To show that that it is possible to do an analysis on link adaptation with this tool and to also give an example how to extend testing even further, we made our own analysis on the link adaptations BLER target. This way we could also validate that our tool could handle large amount of data and that you can compare different data with each other. To be able to show that our tool is usable for Ericsson IODT right now we let IODT test engineers do tests on Link adaptation and HARQ with our tool and evaluate it with their current methods. 
\exhibitpageafterabstract 
\end {document}
