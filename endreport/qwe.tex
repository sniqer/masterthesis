\documentclass[cropmarks, frame, english]{idamasterthesis}

\author{Paul Nedstrand \& Razmus Lindgren}
\titleenglish{Test Data Post-Processing and Analysis of LA \& HARQ}
\titleswedish{Svensken Titel}
\publicationmonth{THEMONTH}
\publicationyear{2015}
\isbn{ISBN}
\thesisnumber{THESISNUMBER}
\thesisyearnumber{THESISYEARNUMBER}
\dateofpublication{\today}
\supervisor{Ola Leifler}
\examiner{EXAMINER}
\degreesubject{Engineering}
\supportedby{SUPPORTEDBY}

\newcommand{\abbrlabel}[1]{\makebox[3cm][l]{\textbf{#1}\ \dotfill}}
\newenvironment{abbreviations}{\begin{list}{}{\renewcommand{\makelabel}{\abbrlabel}}}{\end{list}}


\abstract{%
\S\ This Master thesis involves developing a lightweight analyser that produce statistics from the communication between a cell phone and a E-UTRAN base station.  
The analyser tool will produce graphs with information about the correlation between a signal throughput and the interference in the channel that the signal is sent through. 
From the statistics produced by the analyser tool, the testing personal at Ericsson can more easily deduce where the interference in signals arises from. 
}

%\layout

\begin{document}

\begin{abbreviations}
\item[US] United States
\item[EU] European Union
\item[Gvmt] Government
\end{abbreviations}


\makeintropages


\chapter{Introduction}

\section{Motivation}



The purpose with this master thesis is to help Long Term Evolution Interoperability Development Testing (LTE IODT) Data analysis. LTE IODT wants to automatically generate analysis of Link Adaptation (LA) and Hybrid Automatic Repeat Request (HARQ) tests where we sweep through Signal to Interference plus Noise Ratio (SINR) for different channel models. The LTE IODT lab test logs gives a unique opportunity to look into detailed behaviour of link and rank adaptation for both downlink and uplink. \newline


Ericsson needed a tool for allowing the testers to better analyse the performance in UE-to-eNodeB implementations. The performance in this sense is whether Ericsson and/or their customer had implemented their system according to the minimum criterion of the specification for said implementation or not. To be able to judge where in the implementation the loss of performance had occurred, Ericsson needed a tool to plot the values of signal-variables that are most affected by HARQ and LA, so that their own personnel could use that information to further analyse the algorithms. To be able to plot relevant values for debugging we first had to gain vast knowledge about how LTE, HARQ and LA. \newline


The visualization tool which we used in this project was an Ericsson internal project called Logtool. We implemented our program as a plugin project to Logtool. The Logtool project is built upon the eclipse framework and we also developed the plugin in eclipse development enviroment in Java. 
\section{Background}
The following section is to provide with information useful to understand the contents of the report. This chapter contains explenation of Link Adaptation, Modulation, Coding and code rate. 
\subsection{Link Adaptation}
Link Adaptation is a way to enhance the performance in systems with wireless signals yes dependent on the channel condition the modulation scheme and code rate is changed. The better channel condition the higher modulation scheme and higher code rate [source on this]?. The modulation scheme used in the LTE systems are QPSK, 16QAM and 64QAM. The codes used is QPP (quadrature polynomial permutation) turbo codes [source on this]?. When data are/is sent from a base station to a UE (DL) the UE will report a CQI (channel quality indicator) value to the base station indicating how good the channel is. CQI can take values from 0 to 15 (4 bits) where 0 represent a very bad channel condition and 15 a very good one. Out from this value The EnodeB decides a MCS (modulation and coding scheme). For downlink MCS can take values between 0-28 and uplink 0-22. Each one of this value represent a Modulation scheme and code rate. 

\subsection{HARQ algorithm}
skriver vi something here eller ar denna section bara massa skit

\subsection{Modulation}
A modulation scheme is a way to map digital bits to analog signals in wireless systems. it is a way to represent the bits in the air.
There are different modulation schemes and the ones that are use in LTE are 4QAM (QPSK), 16QAM and 64 QAM. The signals are modulated in the following way 

(bild pa 4QAM 16QAM and 64QAM over i-phase and q-phase). 
 
The I-phase is basically a Sinus wave and Q-phase is a Cosine wave. This way orthogonality occur. What the points on the two axises represent is the amplitude. So in the QPSK case. What the different signals that actually will be is
is A*sinus(f*t), A*cos(f*t), -A*sinus(f*t) and -A*cos(f*t). So in this case the it fits 2 bits in each signal. I.e e.g signal 1 = 00, signal 2 = 01, signal 3 = 10, signal 4 = 11. In 16QAM and 64QAM each signal point is represented by 4 and 6 bits respectively.

The signals in LTE are modulated with an IQ-modulator and decoded with and IQ-Demodulator

\subsection{Coding and code rate}
A coding is a way to create redundancy in the bits that are send. The data will consist of real data bits and coded bits. This way you are able to correct bits that are wrongly recieved. The more coded bits you have in your message the more errors you can detect and correct, but the slower data rate you will have. The code rate states how many bits that are coded in a message. the Code rate is between 0 and 1 and is simply the ratio between #real bits to #real bits + #coded bits. 
\\ \\
example: if we have a code rate of 0.73 we have 73% real bits in the message and 27% coded bits. 





 


TODO: write about all web pages and libraries which contained information about LTE

The analyzer should be able to plot several graphs, save them and read them again such that the user can compare different traces.

\section{Goals and Methodology}

Our task from Ericsson was to develop a lightweight analyzer tool that:

 \begin{itemize} 
 \item simply produces statistics
 \item Handle multiple input sources
 \end{itemize} 

\setlength{\parindent}{0cm} We also had the following criteria: 
 
  \begin{itemize} 
 \item Study and understand the 3GPP standards and Ericsson Research
 \item Analyzfrom processing for the final graph
 \item Evaluate a suitabe the data le tool for the processing of data, e.g. MATLAB or other tools
 \item Capability to correlate the graph to logs in order to facilitate troubleshooting
\end{itemize}

TODO: write about why and how we did a focus group

\section{Thesis Outline}
This thesis is divided into the following Parts
\begin{itemize}
	\item we planned to do a analyzer tool and then we could do an analysation of link adaptation with this tool
	\item part 2 we implemented stuff here
	\item part 3 we reviewed stuff here
\end{itemize}

\chapter{The analysation tool}
\chapter{Analysation of Link Adaptation}

\section{problem formulation}
The purpose of this master thesis is to help LTE IODT Data Analysis to see how well the LTE system is performing when the HARQ (Hybrid Automatic Repeat Request)and LA (Link Adaptation) is activated. To perform such an analysis we needed to gather vast knowledge regarding how signals are treated in LTE and how the protocols implemented are constructed. With this tool we shall help LTE IODT be able to different analysises on our tool. 

\section{The analysis}
Paul skriver nedanstaende \\ \\
what our analysis is intended to do is to look at how well performed the different mcs values are in the AMC (adaptive modulation control) protocol. this is dependent on the different CQI values in downlink, and from this value, the BLER (block error rate) value the enodeB choses a MCS 5-bits. What our analysis is intended to study is to see of the optimal MCS is chosen according to the sinr value. This way we can see if some mcses might be redundant och that some mcs value should be at other cqi reports. 

Uplink: \newline
Data is sent from the UE to the enodeB. The EnodeB is calculating SINR and from this value and block error rate (and maybe something more) the enodB decides which MCS the UE shall send at. MCS in uplink varies between 0-22, where mcs = 0 in the worst channel conditions (lowest SINR's) and mcs = 22 in the highest SINR's. 

Downlink: \newline
Data is sent from the enodeB to the UE. When the data is sent from the enodeB the UE is calculating a CQI value (0-15) which represent the channel condition. 0 is the worst channel condition and 15 is the best channel conditions. this value is sent back to the enodeB and from this value and some other parameters (BLER) an MCS is chosen. In this case this value varies between 0-28 where 0 is the modulation and coding scheme for the worst channel conditions and 28 is the modulation schemes for the best channel conditions.

what we have looked at is both uplink and downlink. in the downlink we did 30 different simulations. 29 simulations where we have hardcoded the enodeB to run at a specific MCS, 0-28 and one where did not hardcode a mcs. We compared all these curves to each other look at which SINR / CQI value they intersect each other and if the hardcoded mcs value were higher than the actual value it had.

\section{Mainline}

text...


\chapter{Closing}

text...

\end{document}
